Kongsberg Gruppen har samarbeidet med studenter gjennom flere år på et prosjekt de kaller \emph{LocalHawk}. Prosjektet går ut på å utvikle et fly som kan dekke lokale sportsbegivenheter som foregår i løyper av litt større hastighet og skala enn ballsport og friidrett. Et eksempel på hva de ser for seg er bilsport, der den kommersielle dekningen i dag er begrenset til å bruke kameraer plassert langs banen eller helikoptre.

Tidligere år er det blitt utviklet et fly som passer spesifikasjonene som er nødvendige til å dekke et billøp, og flyet er blitt utstyrt med programvare som gjør at det enten kan styre seg selv eller følge preprogrammerte løyper. Vår oppgave skulle altså være å implementere bildefunksjonaliteten i flyet, som både skulle kunne detektere og følge etter objekter og sende en videostrøm fra flyets kamera ned til en basestasjon for TV-bruk.

\subsection{Tidligere erfaring}

Gruppa hadde lite erfaring med bildegjenkjenning da denne oppgaven ble valgt. Etter et dypdykk i det tekniske virket bildegjenkjenning så det ut til at det faglige grunnlaget var allerede til stede hos gruppemedlemmene.

\textcolor{red}{Konstruksjonen av riggen var det god oppslutning rundt, både faglig og erfaringsmessig.}\colorbox{yellow}{Må endres}
