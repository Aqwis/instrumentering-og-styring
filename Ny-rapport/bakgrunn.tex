Kongsberg Gruppen har samarbeidet med studenter gjennom flere år på et prosjekt de kaller ``LocalHawk''. Prosjektet går ut på å utvikle et fly som kan dekke lokale sportsbegivenheter som foregår i løyper av en litt større hastighet og skala enn ballsport og friidrett. Et eksempel på hva de ser for seg er bilsport, der den komersielle dekningen i dag er begrenset til kameraer langs banen eller helikoptre. 

Tidligere år har det blitt utviklet et fly som passer spesifiksjonene til å dekke et billøp, og det er blitt utstyrt med programvare som gjør at det kan styre seg selv, eller følge preprogrammerte løyper. Vår oppgave skulle være å implementere kameraet i flyet, som både skulle stå for evnen til å følge etter objekter, samt å sende en videostrøm av hva flyet så ned til en basestasjon.

\subsection{Tidligere erfaring}

Gruppa hadde lite erfaring med bildegjenkjenning da vi valgte denne oppgaven, men bildegjenkjenning virket som noe vi kunne klare å sette oss nok inn i til å lage dette systemet. Det faglige grunnlaget som var nødvendig var allerede der, og alle var interessert i å utvikle noe håndfast.

Konstruksjonen av riggen var det god oppslutning rundt, både faglig og erfaringsmessig. 