I denne rapporten beskrives implementasjonen av funksjonalitet for å identifisere og følge et objekt med et kamera montert til en bevegelig mekanisk rigg. Oppgaven er basert på Kongsbergs omfattende \emph{LocalHawk}-prosjekt, der målet er å utvikle et autonomt fly som skal kunne følge en bil på en racerbane. Ved bruk av kameraet og et system for objektgjenkjenning, benyttes den mekaniske riggen til å holde objektet i sentrum av bildet. Prototypen beskrevet i denne rapporten benytter seg av en Arduino og tre analoge servoer av typen PowerHD til å kunne følge instruksjoner gitt av bildeprosesseringsprogramvaren, skrevet i C++ ved bruk av biblioteket OpenCV.

Systemet har vist seg å fungere i et testmiljø, men krever en del videre arbeid for at det skal kunne brukes i en reell sammenheng. Den utviklede implementasjonen detekterer og fanger opp et ensfarget, enkelt objekt. Å følge en bevegelig bil sett fra et fugleperspektiv, som verken er ensfarget eller alltid betraktes fra samme avstand eller retning, er betydelig mer komplisert. Videre anbefales det å i stedet for Arduino benytte eksempelvis Raspberry Pi, som har større evne til å samtidig prosessere data fra et gyrometer og et kamera under flyturen. 

%Konklusjon nedenfor 
\iffalse
Planleggingsfasen av prosjektet var preget av at det ble gjort en litt for optimistisk ved vurdering av hvor mye som var tilgjengelig innenfor rammene av faget. Oppgaven som ble presentert av Kongsberg gruppen hadde mye relevant for faget, samt at det hørtes meget spennende ut. Arbeidet med prosjektet har ført til alle på gruppa har lært mye nytt, både om arbeidet ved ukjente teknologier og kompleksiteten ved disse.

Prototypen som ble laget selekterer ut et objekt fra en videostrøm og deretter dirigerer en mekanisk rigg med et kamera, slik at objektet hele tiden holder seg i sentrum. Prototypen ble bygget opp av tre analoge servoer av typen PowerHD, som benytter seg av PDM-signal for justering av retning. For modellering av bildene ble C++ brukt i kombinasjon med OpenCV, noe som viste seg å fungere utmerket til dette formålet. 

Det gjenstår en del arbeid før det som er påbegynt, kan brukes i sammenheng med flyet som Kongsberg gruppen har sett for seg. Bildegjenkjenning er et komplekst område i seg selv, og egner seg nok som et eget prosjekt uavhengig av om LocalHawk skal inkluderes. I tillegg vil det være fornuftig å benytte seg av for eksempel en Raspberry PI, med servoer \textcolor{red}{designet} for denne. Dette vil bidra til støtte for multi-threading, noe som igjen gjør det mulig å benytte seg av et for eksempel et gyrometer i tillegg til kamerafeeden.  
\fi