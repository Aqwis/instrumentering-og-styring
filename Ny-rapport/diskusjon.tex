\begin{itemize}
\item 
Droppet gyro.
\item 
Tungt kamera.
\item 
Digitale servoer ville holdt kameraet roligere.
\item I demostrasjone  ble alle også servo 3 brukt for å følge objekt.

\item Hva kunne vært fordelen med å gå over til Raspberry PI med sine servoer

\item diskutere kompleksiteten ved bildegjenkjenning. 

\item diskutere behandling av bilder i flyet med egen gpu

\item diskutere utfordringene med å følge bilen med flyet, med tanke på svingene på banen og flyets krenging.

\item hva de neste steg i prosessen blir, for å komme frem til et brukbart produkt. 
\end{itemize}

 
\subsection{Mekanisk instalasjon}

Kameraet som ble brukt var et billig 300K piksels webkamera. Dette til tross for at Kongsberg sende et kamera som var tiltenkt brukt i LokalHawk flyet. Grunnen er at konstruksjonen av riggen ble gjort før Kongsberg sendte sitt kamera og ble laget med tanke på et mindre kamera og da kameraet skulle monteres på riggen viste det seg at det var for stort og tungt. Hadde kameraet fra Kongsberg vært skaffet tidligere kunne riggen blitt bygget større med større og sterkere servoer. Dermed ble et nytt kamera kjøpt, dette kameraet er betydelig mindre og lettere. Selvom det har dårligere spesifikasjoner er det godt nok til å vise at bildegjenkjenningen virker som den skal.

På grunn av liten tid til testing mot slutten av prosjektet ble også servomotor 3 benyttet til å følge objekter selvom denne egentlig bare skulle brukes til å heve å senke riggen. Dette fordi det var enklere å implemnetere en algoritme for å følge objekt med alle servoene på den lille tiden dom var igjen. Det viste seg at servo 3 kunne vibrere så lenge bommen ``hvilte'' på armen til denne. Det var bare en liten vibrasjon, men den kan ha innvirkning hvis objektet er langt fra kameraet og bildet er zoomet inn. Denne vibrasjonen er nok mest sannsynlig på grunn av oppdateringshastigheten til servoen. Som bevnt tidligere er denne på 50Hz og vibrasjonen kunne vært enda mindre ved bruk av digitale servoer som har høyere oppdateringshastighet.   