\begin{itemize}
\item diskutere kompleksiteten ved bildegjenkjenning. 

\item diskutere utfordringene med å følge bilen med flyet, med tanke på svingene på banen og flyets krenging.

\item hva er ulempen / fordelen med å bruke USB? Det er nenvt i teori, og må kommenteres. 
\end{itemize}

\subsection{Implementasjonen}
 
\subsubsection{Mekanisk installasjon}


På grunn av liten tid til testing mot slutten av prosjektet ble også servomotor 3 benyttet til å følge objekter selvom denne egentlig bare skulle brukes til å heve å senke riggen. Dette fordi det var enklere å implementere en algoritme for å følge objekt med tre servoer. Det viste seg at servo nr. 3 kunne vibrere så lenge bommen ``hvilte'' på armen til denne. Det var bare en liten vibrasjon, men den kan ha innvirkning hvis objektet er langt fra kameraet og bildet er zoomet inn. Denne vibrasjonen er nok mest sannsynlig på grunn av oppdateringshastigheten til servoen. Som nevnt tidligere er denne på 50Hz og vibrasjonen kunne vært enda mindre ved bruk av digitale servoer som har høyere oppdateringshastighet.  

\subsubsection{Gyrometer}
Gyrometeret i systemet skulle bli benyttet til å simulere flyets krengninger. Det kunne da bli demostrert hvordan riggen ville oppføre seg mens den var i bevegelse. Drivere til gyrometeret fantes ferdig til Arduino, og det var få problemer å få det til å virke. Det oppsto derimot problemer da servoene skulle implimenteres i samme program. Det ble konflikt mellom de to bibliotekene fordi begge benytter seg av samme timer på Arduinoen. Siden det ikke er multithreading på Arduino, er det ikke mulig å få et synkront program. Gyro-biblioteket fungerte også bra til Raspberry Pi. Gyrometeret ble valgt bort da det var problematisk å få servoene til å virke på Raspbery Pi. En løsning med én enkelt Arduino og tre servoer servoer ble derfor benyttet. 


For systemets helhet, er ikke gyrometeret vitalt. I spesifikasjonene til Kongsberg blir det presisert at systemet allerede inneholder et gyrometer. Ved en fullstending implementasjon vil kan denne bli brukt. Et eget gyrometer resulterer i et system som er uavhengig av den eksisterende hardwaren på Local Hawk. Det vil da bli lettere å overføre systemet til et annet bruksområde. 


\subsubsection{Bildegjenkjenning}

Det tok ikke veldig lang tid før det måtte erkjennes at bildegjenkjenningen var mye mer utfordrende enn først antatt. Selv en oppgave som å følge et ensfarget objekt medfører ganske komplekse algoritmer. Små endringer i lysforhold kan være nok til å endre kvaliteten på et bilde nok til at objektet plutselig faller gjennom filteret i programmet eller at forstyrrende elementer fra bakgrunnen kommer gjennom filteret og fører til en mengde støy som gjør at programmet ikke finner noe klart objekt å forfølge.

Målet for bildegjenkjenningen var å klare å følge meget komplekse objekter i varierende terreng. En bil brukt i bilsport vil ofte ha en eller flere farger som grunnfarge, i tillegg har vil det være et udefinert antall klistremerker og påmalte kjennemerker som annonser, navn og annet. I utviklingen av algoritmene ble det aldri tid til å komme forbi steget der ensfargede objekter i varierte bakgrunner skal benyttes. Dette fungerte nesten utelukkende dersom kontrasten på fargen til objektet var stor nok i forhold til bakgrunnen til at de enkelt kunne skilles fra hverandre.

Bildebehandling er et komplisert fagfelt, og gjenkjenning av kompliserte objekter, som realistiske biler, er en kompleks prosess som krever lang erfaring og en svært stor mengde arbeid. Arbeid med bildegjenkjenning på det nivået som kreves for å skille ut objekter i den grad som ønskes av Kongsberg, var ikke mulig å oppnå i den relativt korte tiden som var tilgjengelig for dette prosjektet.

\subsubsection{Ressursbruk i bildegjenkjenning og bruk i flyet}

Bildegjenkjenning er en tung prosess som krever noe hardware for å utføre operasjonen fortløpende. En relativt kraftig CPU trengs når videostrømmen skal behandles i sanntid. I spesifikasjonene fra Kongsberg \cite{LocalHawkPDF} var det ikke spesifisert noen form for hardware som støtter den mengden prosessering som er nødvendig for gjenkjenningen. I samtaler med Kongsberg kom det frem at de vurderte å utstyre flyet med en egen GPU, for å utføre bildegjenkjenning. 

GPU har i nyere tid kommet til en formfaktor og energibruk som egner seg godt til bruk for et prosjekt som dette. OpenCV har egne pakker som inkluderer støtte for bruk av GPU for behandling av bilder, og kodebasen som er utviklet i dette prosjektet kan derfor i stor grad benyttes selv om denne løsningen blir brukt.

\subsubsection{Nedskalering av oppgaven}
Arbeidet med prosjektet kom fort igang, oppgaven var tidlig valgt ut og teknologiene som skulle benyttes bestemt. Bildegjenkjenningen hadde vanskeligheter i begynnelsen, særlig med biblioteket \emph{mexopencv}. Problemer med installasjon førte resulterte i at programmeringen aldri var på et stadie der vi kunne begynne å bruke Matlab. Det tok noen uker med forsøk før vi bestemte oss for å bytte til C++. Kongsberg gruppen ble informert om dette, og de var enige i at det var en god løsning. Resultatet av endringen førte til at vi innen få dager hadde vi utviklet en enkel løsning for bildegjenkjenning.

Rundt tiden da samarbeidsavtalen ble revidert ble det klart at gruppa ikke kunne komme i mål med hele oppgaven slik den var fremstilt av Kongsberg. Prosjektets tilstand ble diskutert innad i gruppen, og i samråd med faglærer ble en prototype av bildestabiliseringsriggen bestemt til å være en bedre proporsjonert oppgave for den gitte tidsrammen.

\subsection{Videre arbeid}
Siden prototyperiggen presentert i denne oppgaven er for liten til å bære kameraet Konsberg ønsker å bruke i sitt LocalHawk, er en naturlig videre føring av dette prosjektet å bygge en ny rigg og forbedret rigg med større servoer og materialer mer egent vær og vind. Det neste steget vil kunne være å bruke et gyrometer, som det ble forsøkt i dette prosjektet, til å stabilisere denne riggen slik at den blir mest mulig uavhengig av flyets bevegelser. 

Bildegjenkjenningen trenger fremdeles en god del arbeid før den kan brukes i faktiske forhold. Som tidligere nevnt innebærer dette nesten et selvstendig prosjekt hvor fokuset er på bearbeidingen av bilder og integrasjon av denne teknologien mot en innebygget GPU i flyet. Prototypen som er utviklet i dette prosjektet viser tydelig hvordan oppgaven med bildestabilisering kan løses som en isolert oppgave. Resultatet tar ikke for seg hvordan bilens endrede kjøreretning vil påvirke flyets krengning, og deretter hvordan dette dekkes av kameraet.

Kongsberg så for seg en løsningen som et hyllevareprodukt som kan plasseres under flyet. En videreføring av implementasjonen vil være å lage et eget kretskort med gyro, CPU og GPU. Til dette formålet kan ARM-prosessor benyttes, som også har innebygget GPU med god ytelse. Et felt som ikke blir dekket i denne rapporten er signaloverføringen fra flyet og til en bakkestasjon. Til dette formålet kan WiFi eller GSM-nettet bli benyttet. Et moment som må bli tatt hensyn til, er at mange tilskuere på banen kan påvirke signaloverføringen.

Beslutningen om nedskalering av prosjektet medførte å påvise at riggen i prinsippet klarer å følge etter objekter. I dag er stabiliseringen litt hakkete, noe som bare betyr at det kan gjøres med for å forsøke å enten forutsi hvilken vei objektet sannsynligvis beveger seg eller å la kameraet følge litt etter slik at banen kameraet skal følge kan finstilles mer.

