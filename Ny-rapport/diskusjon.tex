\begin{itemize}
\item 
Droppet gyro.
\item 
Tungt kamera.
\item 
Digitale servoer ville holdt kameraet roligere.
\item I demostrasjone  ble alle også servo 3 brukt for å følge objekt.

\item Hva kunne vært fordelen med å gå over til Raspberry PI med sine servoer

\item diskutere kompleksiteten ved bildegjenkjenning. 

\item diskutere behandling av bilder i flyet med egen gpu

\item diskutere utfordringene med å følge bilen med flyet, med tanke på svingene på banen og flyets krenging.

\item hva de neste steg i prosessen blir, for å komme frem til et brukbart produkt. 
\end{itemize}

\subsection{Gyrometer}
Gyrometeret i systemet skulle bli benyttet til å simulere flyets krengninger. Det kunne da bli demostrert hvordan riggen ville oppføre seg mens den var i bevegelse. Drivere til gyrometeret fantes ferdig til Arduino, og det var få problemer å få det til å virke. Det oppsto derimot problemer da servoene skulle implimenteres i samme program. Det ble konflikt mellom de to bibliotekene fordi begge benytter seg av samme timer på Arduinoen. Siden det ikke er multithreading på Arduino, er det ikke mulig å få et ?kongruent? program. Gyro-biblioteket fungerte også bra til Raspberry Pi. Men gyrometeret ble valgt bort da det var problematisk å få servoene til å virke på Raspbery Pi. 

For systemets helhet, er ikke gyrometeret vital. I spesifikasjonene til Kongsberg blir det presisert at systemet allerede inneholder et gyrometer. Ved en fullstending implementasjon vil denne bli brukt. Fordelen ved å ikke bruke gyrometeret, er at implimentasjonen blir lettere. Ved bruk av gyrometer, må pådraget til servoene ta hensyn til hviklen vinkel riggen står i. 

\subsection{}
