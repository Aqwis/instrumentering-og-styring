Vår gruppe er sammensatt av studenter med en tverrfaglig studiebakgrunn. I løpet av de første møtene ble det fort klart at gruppa som en samlet enhet har bred kompetanse innenfor både konstruksjon, og programmering av roboter. Studieretningen for det respektive teamet er som følger:

\begin{description}
	\item[Håvard Slettvold] Informatikk
	\item[Adrian Finvold] Kjemiingeniør
	\item[Raymond Dyngeseth Selvik] Kybernetikk
	\item[Endre Larsen] Elektronikk
	\item[Trygve Bertelsen Wiig] Industriell matematikk
\end{description}

Kongsberg var en av bedriftene som presenterte prosjekter for vår landsby. De presenterte LocalHawk og gruppen var av den generelle oppfatningen om av at det virket som et spennende og utfordrende prosjekt. 

Kongsberg ønsket at det skulle utformes et bildestabiliseringssystem for autonome droner, som i deres eksempel skulle brukes til å filme billøp. Bildestabiliseringen skulle foregå med en mekanisk rigg som sørget for å holde ønskede objekter inne i skjermbildet mens flyet manøvrerer. I tillegg til mekanisk stabilisering av skjermbildet var det også nødvendig med et program som gjenkjente objekter i skjermbildet, og ga instrukser til både riggen og flyet. Informasjonen skulle benyttes om hvordan det mekaniske måtte oppdateres for å vedlikeholde dekningsgraden av objektet. 

Oppstartsfasen av samarbeidet baserte seg på planlegging av løsninger og utvekslede synspunkter innad i gruppen. Gruppen angrep problemstillingen ved å benytte seg av en prototype med Arduino, Raspberry Pi og enkle servoer som den mekaniske løsningen. I presentasjonen av prosjektet kom det frem at Kongsberg ønsket å ha kodebasen sin i Matlab. Beslutningen ble derfor å programmere bildegjenkjenningen med OpenCV i Matlab, hvor mexopencv ble benyttet for å knytte sammen OpenCV og Matlab.