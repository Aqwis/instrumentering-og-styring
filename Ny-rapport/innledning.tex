Vår gruppe er sammensatt av studenter fra mange relevante områder for vår landsby. I løpet av de første møtene ble det fort klart at gruppa som en samlet enhet har bred kompetanse innenfor både konstruksjon og programmering av roboter. Fordelingen er som følger:

\begin{description}
	\item[Håvard Slettvold] Informatikk
	\item[Adrian Finvold] Kjemiingeniør
	\item[Raymond Dyngeseth Selvik] Kybernetikk
	\item[Endre Larsen] Elektronikk
	\item[Trygve Bertelsen Wiig] Industriell matematikk
\end{description}

Kongsberg Gruppen var en av bedriftene som presenterte prosjekter for vår landsby, og da de presenterte LocalHawk var den generelle oppfatningen i gruppa at det virket som et spennende og utfordrende prosjekt. 

Ønsket fra Kongsberg Gruppen var at det skulle utformes et bildestabiliseringssystem for autonome droner, som i deres eksempel skulle brukes til å filme billøp. Bildestabiliseringen skulle foregå med en mekanisk rigg som sørget for å holde ønskede objekter inne i skjermbildet mens flyet manøvrerer. I tillegg til mekanisk stabilisering av skjermbildet var det også nødvendig med et program som gjenkjente objekter i skjermbildet og ga instrukser til både riggen og flyet om hvordan alt det mekaniske måtte oppdateres for å vedlikeholde dekningsgraden av objektet. 

Her sa Raymond og Endre at de var interessert i å jobbe med den mekaniske riggen. Dette virket midt i blinken, da begge har særdeles relevante fagområder i sine studier. Trygve og Håvard mente at de kunne klare å ordne med bildegjenkjenningen, Trygve har fra før en del erfaring med programmering utover studiene og for Håvard var det en faglig fulltreffer.

Etter noen møter hvor vi diskuterte løsninger innad på gruppa og utvekslet synspunkter med Kongsberg begynte vi med en prototype med Arduino, Raspberry Pi og enkle servoer som den mekaniske løsningen og OpenCV og C++ for programmeringen.