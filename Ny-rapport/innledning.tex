Gruppen er sammensatt av studenter med en tverrfaglig studiebakgrunn. I løpet av de første møtene ble det fort klart at gruppa som en samlet enhet har bred kompetanse innenfor både konstruksjon, og programmering av roboter. Studieretningen for det respektive teamet er som følger:
%
\begin{description}
	\item[Håvard Slettvold] Informatikk
	\item[Adrian Finvold] Kjemiingeniør
	\item[Raymond Dyngeseth Selvik] Kybernetikk
	\item[Endre Larsen] Elektronikk
	\item[Trygve Bertelsen Wiig] Industriell matematikk
\end{description}
%
Kongsberg gruppen (heretter Kongsberg) var en av bedriftene som presenterte prosjekter for landsbyen. Gjennom flere år har de samarbeidet med studenter på et prosjekt de kaller \emph{LocalHawk}. Prosjektet går ut på å utvikle et fly som kan dekke lokale sportsbegivenheter som foregår i løyper i litt større hastighet og i en større skala enn ballsport og friidrett. Et eksempel på hva de ser for seg er bilsport, der den kommersielle dekningen i dag er begrenset til å bruke kameraer plassert langs banen eller bruk av helikoptre. 

Kongsberg ønsket at det skulle utformes et bildestabiliseringssystem for autonome droner, som i deres eksempel skulle brukes til å filme billøp. Bildestabiliseringen skulle foregå med en mekanisk rigg som sørget for å holde ønskede objekter inne i skjermbildet mens flyet manøvrerer. I tillegg til mekanisk stabilisering av skjermbildet var det også nødvendig med et program som gjenkjente objekter i skjermbildet, og ga instrukser til den mekaniske riggen. Informasjon fra det samme programmet skulle også brukes til å gi henvisninger til flyet om endringer i den planlagte banen som var nødvendig for å vedlikeholde dekningsgraden av objektet. Den generelle oppfatningen til gruppen var at det virket som et spennende og utfordrende prosjekt. 

Tidligere år er det blitt utviklet et fly som passer spesifikasjonene som er nødvendige til å dekke et billøp, og flyet er blitt utstyrt med programvare som gjør at det enten kan styre seg selv eller følge preprogrammerte løyper. Vår oppgave skulle altså være å implementere bildefunksjonaliteten i flyet, som både skulle kunne detektere og følge etter objekter og sende en videostrøm fra flyets kamera ned til en basestasjon for TV-bruk.

Oppstartsfasen av samarbeidet baserte seg på planlegging av løsninger og utvekslede synspunkter innad i gruppen. Gruppen angrep problemstillingen ved å benytte seg av en prototype med Arduino, Raspberry Pi og enkle servoer som den mekaniske løsningen. I presentasjonen av prosjektet kom det frem at Kongsberg gruppen ønsket å ha kodebasen sin i Matlab. Beslutningen ble derfor å programmere bildegjenkjenningen med OpenCV i Matlab, hvor mexopencv ble benyttet for å knytte sammen OpenCV og Matlab.