Planleggingsfasen av prosjektet er preget av at det ble gjort en litt optimistisk vurdering av hvor mye vi hadde tid til innenfor rammene til faget. Oppgaven som ble presentert av Kongsberg hadde mye relevant for faget, samt at det hørtes meget spennende ut. Arbeidet med prosjektet har ført til alle på gruppa har lært mye nytt, både om prosessen rundt prosjektarbeid og fallgruver, som for eksempel ukjente teknologier.

Tatt i betraktning problemene gruppen har hatt med utviklingen er vi fornøyde med resultatet. Prototypen klarer å velge et objekt fra en videostrøm og dirigere en mekanisk rigg med et kamera slik at objektet hele tiden holder seg i sentrum.

Det gjenstår en del arbeid før det som er påbegynt i dette prosjektet kan brukes i sammenheng med flyet som Kongsberg har laget. Bildegjenkjenningen er et stort prosjekt i seg selv, og egner seg nok som et eget prosjekt uavhengig av LocalHawk.