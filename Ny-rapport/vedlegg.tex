\subsection{Vedlegg A: Arduino C++ kode}

\begin{frame}
\frametitle{Styring av servorigg - Arduino}
\lstset{language=C++,
				inputencoding=utf8,
                keywordstyle=\color{blue},
                stringstyle=\color{red},
                commentstyle=\color{green},
                morecomment=[l][\color{magenta}]{\#}
}
\begin{lstlisting}

//På servo er brun ledning jord, 
// rød ledning kobles til 5V 
// orange er signal og kobles til pwm port

#include <Servo.h>

//Definerer servoobjekter
Servo servo_1; 
Servo servo_2;
Servo servo_3;

void setup(){
  servo_1.attach(9);  //Servo 1 er koblet til port 9
  servo_2.attach(6);  //Servo 1 er koblet til port 6
  servo_3.attach(3);  //Servo 1 er koblet til port 3
  Serial.begin(9600);
}

void loop(){
  
  static int ut = 0;
  
  if (Serial.available()){
    char abc=Serial.read(); 
//abc skal se ut som dette 111a222b333c hvor tallene foran a er vinkelen 
//til servo_1 tallene mellom a og b er vinkelen til servo_2 osv.
//Det er også mulig å sende komando til en eller to av servoene av gangen også, ved f.eks "100a" eller "50a120c"
    
    //Setter verdiene til servoene og sender de til servoene.
    switch(abc) {
      case '0'...'9':
        ut = ut*10 + abc -'0';
        break;
      case 'a':
        servo_1.write(ut);
        ut = 0;
        break;
      case 'b':
        servo_2.write(ut);
        ut = 0;
        break;
      case 'c':
        servo_3.write(ut);
        ut = 0;
        break;
    }
  }
  
}  

\end{lstlisting}
\end{frame}