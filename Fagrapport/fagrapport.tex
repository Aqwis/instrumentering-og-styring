\documentclass[a4paper, 11pt]{article} 
\usepackage[T1]{fontenc} % Vise norske tegn 
\usepackage[utf8]{inputenc} % For å kunne skrive norske tegn 
\usepackage{graphicx} % For å inkludere grafikk 
\usepackage{amsmath,amssymb,amsfonts,amsthm} % Ekstra matematikkfunksjoner 
\usepackage{siunitx} % gir \SI{små tall}{enhet} f.eks. \SI{5}{cm} 
\usepackage{refstyle} % gir eqref, figref, tabref
\usepackage{lscape}
\usepackage{epstopdf}
\usepackage{tabulary}
\usepackage[superscript]{cite}
\renewcommand{\thefootnote}{\alph{footnote}}
\usepackage{fourier}
\usepackage{pdfpages}
\usepackage{lipsum}
\usepackage{float} % gir H som plassering, HERE 
\usepackage{url} % Skriver URL-er fint.
\usepackage{booktabs} % Gir fine tables\usepackage{mathtools} % for bmatrix*, krever pakken texlive-latex3
\usepackage{algorithm} % en float som inneholder algoritme + caption
\floatname{algorithm}{Algoritme}

\usepackage{algpseudocode} % for pseudokode
\usepackage{enumitem} % mer kompakte lister (bruk f.eks. itemize*)
\setlist{nolistsep}
\usepackage{extarrows} % Langt =-tegn. \xlongequal
\renewcommand{\figurename}{Figur} % Norsk Figurtekst
 \renewcommand*\contentsname{Innhold}

\renewcommand{\tablename}{Tabell} 
%%% GRUPPEMEDLEMMER %%%
\newcommand{\HS}{Håvard Slettvold} 
\newcommand{\AF}{Adrian Finvold} 
\newcommand{\TW}{Trygve Bertelsen Wiig} 
\newcommand{\RS}{Raymond Dyngeset Selvik} 
\newcommand{\EL}{Endre Larsen} 
%%%%%%%%%%%%%%%%%%%
\usepackage{epstopdf}
\epstopdfDeclareGraphicsRule{.gif}{png}{.png}{convert gif:#1 png:\OutputFile}
\AppendGraphicsExtensions{.gif}

\newcommand{\hilight}[1]{\colorbox{yellow}{#1}}
\usepackage{color}
\setlength{\parindent}{0.0in} % Paragraph word indentation 
\setlength{\parskip}{0.1in} % Paragraph distance 

\usepackage{titlesec} % Allows customization of titles
\usepackage[hmarginratio=1:1,top=32mm,columnsep=20pt]{geometry} % Document margins
\usepackage[hang, small,labelfont=bf,up,textfont=it,up]{caption} % Custom captions under/above floats in tables or figures
\titleformat{\section}[block]{\Large\scshape\centering}{\thesection.}{1em}{} % Change the look of the section titles
\titleformat{\subsection}[block]{\large}{\thesubsection.}{1em}{} % Change the look of the section titles
\usepackage{fancyhdr} % Headers and footers
\pagestyle{fancy} % All pages have headers and footers
\fancyhead{} % Blank out the default header
\fancyfoot{} % Blank out the default footer
\fancyhead[C]{Instrumentering og styring over Internett - Fagrapport}
\fancyfoot[RO,LE]{\thepage} 
%----------------------------------------------------------------------------------------
%  TITLE SECTION
%----------------------------------------------------------------------------------------
\title{Instrumentering og styring over Internett\\Eksperter i Team\\Fagrapport}
\author{Gruppe 1}
\date{\today}


\begin{document} 

\thispagestyle{fancy} % All pages have headers and footer

%----------------------------------------------------------------------------------------
%  Forside
%----------------------------------------------------------------------------------------
\maketitle 

\vspace*{\fill}
\begin{center}
\AF,\hspace{0.1cm}\HS,\hspace{0.1cm}\TW \\
\RS\hspace{0.1cm}og\hspace{0.1cm}\EL
\end{center}

\clearpage

%----------------------------------------------------------------------------------------
%  Forord
%----------------------------------------------------------------------------------------
 \newpage
\section*{Forord}

%----------------------------------------------------------------------------------------
%  Sammendrag
%----------------------------------------------------------------------------------------
 \newpage
\section*{Sammendrag}


%----------------------------------------------------------------------------------------
%  Table of Content
%----------------------------------------------------------------------------------------
 \newpage

 \tableofcontents
\newpage


%----------------------------------------------------------------------------------------
%  Innledning
%----------------------------------------------------------------------------------------
 \newpage
\section{Innledning}


%----------------------------------------------------------------------------------------
%  Resultat
%----------------------------------------------------------------------------------------
 \newpage
\section{Resultat}

\subsection{Optimalisering av kamerarigg}
Bla bla

\subsection{Bildegjenkjenning}

Kompleksiteten øker proporsjonalt med kompleksiteten i legemet. Flere forsøk viser at det oppstår problemer med gjenkjenning av objekter når vinklene i objektet øker. En illustrasjon av objektene er gitt i figur \ref{fig:platonic}.

\begin{figure}[h]
\centering
\includegraphics[width=4in]{figurer/platonic-solids.gif}
\caption{Illustrasjon av økt kompleksitet ved økende antall hjørner. \cite{platonic} }
\label{fig:platonic}
\end{figure}

%----------------------------------------------------------------------------------------
%  Diskusjon
%----------------------------------------------------------------------------------------
 \newpage
\section{Diskusjon}




%----------------------------------------------------------------------------------------
%  Bibliography
%----------------------------------------------------------------------------------------
\newpage
\bibliographystyle{unsrt}
\bibliography{fagrapport.bib}

%----------------------------------------------------------------------------------------
%  Vedlegg
%----------------------------------------------------------------------------------------
 \newpage
\section{Vedlegg}

\subsection{Kode for bildegjenkjenning (C++)}

\subsection{Kode for styring av rigg (C++)}

\end{document}
